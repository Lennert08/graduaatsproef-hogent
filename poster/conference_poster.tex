\documentclass{hogent-poster}

\usepackage{graphicx}
\usepackage{booktabs}
\usepackage{wrapfig}
\usepackage[dutch]{babel}
% Nieuwe pakketten voor de layout:
\usepackage{tcolorbox} 
\usepackage{pifont} % Voor mooie vinkjes
\usepackage{amssymb} % Voor wiskundige symbolen

% --- CONFIGURATIE VAN DE KLEURBLOKKEN ---
% Een blauw blok voor technische info
\newtcolorbox{techbox}[1]{
    colback=hogent-grey!10, % Lichte achtergrond
    colframe=hogent-blue,   % HOGENT blauwe rand
    title=#1, 
    fonttitle=\bfseries\Large,
    boxrule=2pt,
    arc=5pt
}

% Een oranje blok voor de conclusie (opvallend!)
\newtcolorbox{conclusionbox}[1]{
    colback=hogent-ochre!10, 
    colframe=hogent-ochre,
    title=#1, 
    fonttitle=\bfseries\Large,
    boxrule=2pt,
    arc=5pt
}

% --- GEGEVENS ---
\title{Ontwikkeling van een Generieke Digitale Ondertekenmodule}
\author{Lennert Van Daele}
\supervisor{Tommy Uytterhaegen}
\cosupervisor{Kris De Ridder (IRC Engineering)}
\academicyear{2025-2026}
\institution{Hogeschool Gent}

\begin{document}

\maketitle

\begin{multicols}{3}

    % ================= KOLOM 1 =================
    \section{Situatieschets}
    IRC Engineering ontwikkelt het cloudplatform **Recytix** voor de afvalsector. Klanten beheren hierin contracten en mandaten.
    
    \textbf{\textcolor{hogent-blue}{Het probleem:}}
    \begin{itemize}
        \item Contracten werden afgedrukt, handmatig getekend en weer ingescanned.
        \item Proces was tijdrovend, foutgevoelig en administratief zwaar.
        \item Geen centrale opvolging.
    \end{itemize}

    \section{Doelstelling}
    Het realiseren van een **digitale handtekeningmodule** die naadloos integreert in Recytix.
    
    \vspace{0.5cm}
    \textbf{De Vereisten:}
    \begin{itemize}
        \item[\textcolor{hogent-darkgreen}{\checkmark}] \textbf{Rechtsgeldig:} Identificatie via 2FA (SMS of Itsme).
        \item[\textcolor{hogent-darkgreen}{\checkmark}] \textbf{Gebruiksvriendelijk:} Geen externe software nodig.
        \item[\textcolor{hogent-darkgreen}{\checkmark}] \textbf{Traceerbaar:} Volledige Audit Trail.
    \end{itemize}

    \vspace{1cm}

    % --- TECH STACK IN EEN MOOI BLOK ---
    \begin{techbox}{Technologie-stack}
        De module is gebouwd binnen de bestaande stack:
        \vspace{0.5cm}
        \begin{center}
            \begin{tabular}{ll}
                \toprule
                \textbf{Frontend} & React (TypeScript) \\
                \textbf{Backend} & C\# .NET Web API \\
                \textbf{Database} & PostgreSQL \\
                \textbf{Auth} & Smstools API \& Itsme \\
                \bottomrule
            \end{tabular}
        \end{center}
    \end{techbox}

    % ================= KOLOM 2 =================
    \section{De Oplossing: User Journey}
    De eindgebruiker doorloopt een intuïtief proces.

    \vspace{0.5cm}
    
    % Visuele stappen met kleur
    \textbf{\textcolor{hogent-blue}{1. Notificatie}} \\
    Bij het inloggen ziet de gebruiker direct een 'Toast'-melding: \textit{"U heeft 5 openstaande contracten"}.
    
    \vspace{0.5cm}
    
    \textbf{\textcolor{hogent-blue}{2. Inzage \& Keuze}} \\
    De gebruiker opent het document (PDF-preview) en klikt op "Ondertekenen". Er verschijnt een wizard.

    \vspace{0.5cm}
    
    % SCREENSHOT (SMS.PNG)
    \begin{center}
        \includegraphics[width=0.95\linewidth, frame]{../gradproef/img/sms.png}
    \end{center}

    \vspace{0.5cm}

    \textbf{\textcolor{hogent-blue}{3. Verificatie (2FA)}} 
    \begin{itemize}
        \item \textbf{SMS-flow:} Code via GSM.
        \item \textbf{Itsme-flow:} Bevestiging via App.
    \end{itemize}

    \vspace{0.5cm}

    \textbf{\textcolor{hogent-blue}{4. Afronding}} \\
    Status wordt \textbf{'Getekend'}. Audit trail wordt opgeslagen.

    % ================= KOLOM 3 =================
    \section{Technische Uitdagingen}
    
    % ITSME LOGO (Nu uit commentaar gehaald en mooi gepositioneerd)
    \begin{center}
        \includegraphics[width=0.5\linewidth]{../gradproef/img/itsme_logo.png}
    \end{center}

    Een specifiek struikelblok was de integratie met **Itsme**. De authenticatie-flow vereist strikte veiligheidsprotocollen (HTTPS) en werkt \textbf{niet op localhost}.
    
    \vspace{0.5cm}
    \textbf{\textcolor{hogent-pink}{De Oplossing:}} \\
    Een gedisciplineerde 'Build \& Deploy' cyclus naar de Linux-testservers van Recytix voor elke test.

    \vspace{2cm}

    % --- CONCLUSIE IN EEN OPVALLEND BLOK ---
    \begin{conclusionbox}{Conclusie}
        De doelstellingen zijn \textbf{ruimschoots behaald}.
        \begin{itemize}
            \item \textbf{SMS \& Itsme:} Beide methodes werken naadloos.
            \item \textbf{Schaalbaar:} Generiek en meertalig opgezet.
            \item \textbf{Productieklaar:} Draait succesvol in de cloud.
        \end{itemize}
    \end{conclusionbox}
    
    \vspace{2cm}
    \centering
    \includegraphics[width=0.6\linewidth]{../gradproef/img/recytix_logo.png}

\end{multicols}

\end{document}