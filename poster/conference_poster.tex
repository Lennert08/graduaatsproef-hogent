\documentclass{hogent-poster}

\usepackage{graphicx}
\usepackage{lipsum}
\usepackage{booktabs}
\usepackage{wrapfig}
\usepackage[dutch]{babel}

% --- GEGEVENS (Gebruik de juiste commando's) ---
\title{Ontwikkeling van een Generieke Digitale Ondertekenmodule}
\author{Lennert Van Daele}

% De template gebruikt Engelse termen:
\supervisor{Tommy Uytterhaegen}  % Vervangt \promotor
\cosupervisor{Kris De Ridder (IRC Engineering)} % Vervangt \copromotor

% \projecttype bestaat waarschijnlijk niet in deze template, dus die laten we weg.
% De opleiding/jaar zit vaak standaard in de class of is niet verplicht.
\academicyear{2025-2026}

\begin{document}

\maketitle

\begin{multicols}{3}

    % --- KOLOM 1 ---
    \section{Situatieschets}
    IRC Engineering ontwikkelt het cloudplatform **Recytix** voor de afvalsector. Klanten beheren hierin contracten en mandaten.
    
    \textbf{Het probleem:}
    \begin{itemize}
        \item Contracten werden afgedrukt, handmatig getekend en weer ingescanned.
        \item Dit proces was tijdrovend, foutgevoelig en administratief zwaar.
        \item Geen centrale opvolging van wie wat getekend heeft.
    \end{itemize}

    \section{Doelstelling}
    Het realiseren van een **digitale handtekeningmodule** die naadloos integreert in Recytix.
    
    \textbf{Vereisten:}
    \begin{enumerate}
        \item \textbf{Rechtsgeldig:} Identificatie via 2FA (SMS of Itsme).
        \item \textbf{Gebruiksvriendelijk:} Geen externe software nodig voor de klant.
        \item \textbf{Traceerbaar:} Volledige Audit Trail (Logboek).
    \end{enumerate}

    \section{Technologie-stack}
    De module is gebouwd met de standaard technologieën van IRC Engineering:
    
    \begin{center}
        \begin{tabular}{ll}
            \toprule
            \textbf{Frontend} & React (TypeScript) \\
            \textbf{Backend} & C\# .NET Web API \\
            \textbf{Database} & PostgreSQL \\
            \textbf{Auth} & Smstools API \& Itsme \\ % HIER AANGEPAST: \&
            \bottomrule
        \end{tabular}
    \end{center}

    % --- KOLOM 2 ---
    \section{De Oplossing: User Journey}
    De eindgebruiker doorloopt een intuïtief stappenplan:

    \textbf{1. Notificatie}
    Bij het inloggen ziet de gebruiker direct een 'Toast'-melding: *"U heeft 5 openstaande contracten"*.
    
    \textbf{2. Inzage \& Keuze} % HIER AANGEPAST: \&
    De gebruiker opent het document (PDF-preview) en klikt op "Ondertekenen". Er verschijnt een wizard met keuze voor verificatie.

    % RUIMTE VOOR SCREENSHOT (LATER) OF LOGO
    \begin{center}
        \textit{(Screenshot: Keuzescherm SMS / Itsme)}
    \end{center}

    \textbf{3. Verificatie (2FA)}
    \begin{itemize}
        \item \textbf{SMS-flow:} Gebruiker ontvangt een 6-cijferige code op zijn GSM en voert deze in.
        \item \textbf{Itsme-flow:} Gebruiker bevestigt zijn identiteit via de Itsme-app (Rijksregisternummer check).
    \end{itemize}

    \textbf{4. Afronding}
    Het document krijgt de status \textbf{'Getekend'}. De audit trail wordt weggeschreven in de database.

    % --- KOLOM 3 ---
    \section{Technische Uitdagingen}
    Een specifiek struikelblok was de integratie met **Itsme**. 
    
    De authenticatie-flow van Itsme vereist strikte veiligheidsprotocollen (HTTPS) en werkt \textbf{niet op localhost}.
    
    \textbf{Oplossing:}
    Tijdens de ontwikkeling moest voor elke test van de Itsme-flow de applicatie effectief **gedeployed** worden naar de online testomgeving op de Linux-servers van Recytix. Dit vroeg om een gedisciplineerde 'Build \& Deploy' cyclus. % HIER AANGEPAST: \&

    \section{Conclusie}
    De doelstellingen zijn niet alleen behaald, maar **overtroffen**.
    
    \begin{itemize}
        \item \textbf{SMS \& Itsme:} Oorspronkelijk was enkel SMS voorzien, maar door een vlotte progressie is ook Itsme gerealiseerd. % HIER AANGEPAST: \&
        \item \textbf{Schaalbaar:} De module is generiek opgezet en meertalig (NL/FR/DE).
        \item \textbf{Klaar voor productie:} De oplossing draait succesvol binnen de Recytix-architectuur.
    </itemize}
    
    \vspace{2cm}
    \centering
    % LOGO
    \includegraphics[width=0.7\linewidth]{../gradproef/img/recytix_logo.png}

\end{multicols}

\end{document}