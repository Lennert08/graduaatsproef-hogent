\chapter{Inleiding}
\label{ch:inleiding}

De digitalisering van bedrijfsprocessen is de laatste jaren in een stroomversnelling geraakt. Bedrijven streven naar efficiëntie en willen af van tijdrovende papieren administratie. Binnen het cloudplatform Recytix, ontwikkeld door IRC Engineering, ontbrak echter nog een essentiële schakel: de mogelijkheid om documenten digitaal en rechtsgeldig te ondertekenen.

\section{Probleemstelling}
Recytix is een uitgebreid softwareplatform voor de afval- en recyclagesector. Gebruikers beheren hierin hun contracten, mandaten en orders. Tot op heden moesten deze documenten echter afgedrukt, handmatig ondertekend, gescand en teruggemaild worden.

Dit proces brengt verschillende nadelen met zich mee:
\begin{itemize}
    \item \textbf{Tijdsverlies:} Het manuele proces vertraagt de doorlooptijd van contracten aanzienlijk.
    \item \textbf{Foutgevoeligheid:} Documenten raken kwijt of worden onleesbaar ingescand.
    \item \textbf{Gebruikersonvriendelijkheid:} Klanten ervaren de administratieve rompslomp als een drempel.
\end{itemize}

Er is dus nood aan een geïntegreerde oplossing waarbij de gebruiker een document kan inzien en direct digitaal kan ondertekenen, zonder het platform te verlaten.

\section{Doelstelling}
Het hoofddoel van deze graduaatsproef is het bouwen van een werkend prototype van een digitale handtekeningmodule binnen Recytix.

Concreet zijn de subdoelen:
\begin{itemize}
    \item Het creëren van een algemeen toepasbare oplossing waarmee gebruikers documenten kunnen inzien.
    \item Het implementeren van een ondertekenflow via een SMS-code (Two-Factor Authentication).
    \item Het garanderen van juridische geldigheid door middel van een sluitende audit trail (logboek van acties).
\end{itemize}

\section{Onderzoeksvragen}
Om dit doel te bereiken, staat de volgende centrale onderzoeksvraag centraal:

\textit{"Hoe kan een veilige en juridisch geldige digitale handtekeningmodule geïntegreerd worden in het bestaande Recytix-platform met behulp van .NET en React?"}

Deze wordt ondersteund door de volgende deelvragen:
\begin{enumerate}
    \item Welke technische vereisten zijn nodig om PDF-documenten veilig weer te geven in de browser?
    \item Hoe kan SMS-verificatie technisch geïmplementeerd worden als geldig authenticatiemiddel?
    \item Op welke manier moet de audit trail worden opgeslagen om bewijskracht te garanderen?
\end{enumerate}

\section{Methodologie}
Voor de ontwikkeling wordt gebruikgemaakt van de watervalmethode, aangepast aan de agile werkwijze van IRC Engineering. We starten met een analyse van de vereisten, gevolgd door het ontwerp van de architectuur en de database. Vervolgens wordt de applicatie iteratief ontwikkeld, waarbij eerst de backend-functionaliteit wordt opgezet en daarna de frontend-interface. Tot slot wordt de module uitgebreid getest.

\section{Opzet van de thesis}
In hoofdstuk \ref{ch:analyse} bespreken we de functionele en technische analyse. Hoofdstuk \ref{ch:technologie} licht de gekozen technologie-stack toe. De effectieve realisatie komt aan bod in hoofdstuk \ref{ch:implementatie} (Implementatie). We sluiten af met een conclusie en toekomstperspectieven in hoofdstuk \ref{ch:conclusie}.