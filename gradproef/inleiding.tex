\chapter{Inleiding}
\label{ch:inleiding}

In dit inleidende hoofdstuk schetsen we de context van het project, definiëren we de probleemstelling en formuleren we de doelstellingen. Daarnaast bakenen we de scope van het project af en geven we een overzicht van de structuur van deze scriptie.

\section{Probleemstelling}
\label{sec:probleemstelling}
Het platform Recytix heeft momenteel geen geïntegreerd systeem om documenten digitaal te laten ondertekenen. Voor cruciale documenten, zoals bijvoorbeeld verkoopovereenkomsten of mandaten, is een efficiënt en juridisch sluitend proces essentieel. Het ontbreken van deze functionaliteit zorgt voor administratieve vertragingen en een minder vlotte klantervaring.

Dit project vult die nood in door een specifieke module te ontwikkelen die naadloos aansluit op het bestaande ecosysteem.

\section{Doelstelling}
\label{sec:doelstelling}
Het hoofddoel van deze bachelorproef is het bouwen van een werkend prototype van een digitale handtekeningmodule binnen Recytix. 

Concreet zijn de subdoelen:
\begin{itemize}
    \item Het creëren van een algemeen toepasbare oplossing waarmee gebruikers documenten kunnen inzien.
    \item Het implementeren van een ondertekenflow via een SMS-code (Two-Factor Authentication).
    \item Het garanderen van juridische geldigheid door middel van een sluitende audit trail (logboek van acties).
\end{itemize}

\section{Onderzoeksvragen}
\label{sec:onderzoeksvragen}
Op basis van de probleemstelling en doelstelling formuleren we de volgende centrale onderzoeksvraag:
\begin{quote}
    \textit{Hoe kan een digitale handtekeningmodule met SMS-integratie en audit trail geïntegreerd worden in het Recytix-platform om het contractproces te digitaliseren?}
\end{quote}

\section{Methodologie}
\label{sec:methodologie}
Voor de ontwikkeling van deze applicatie wordt gebruikgemaakt van de volgende technologieën:
\begin{itemize}
    \item \textbf{Frontend:} React voor de gebruikersinterface.
    \item \textbf{Backend:} C\# (.NET) voor de API en logica.
    \item \textbf{Database:} PostgreSQL voor dataopslag en de audit trail.
\end{itemize}

De ontwikkeling verloopt iteratief, waarbij eerst de analyse en architectuur worden opgezet, gevolgd door de implementatie van de backend en vervolgens de frontend.

\section{Opzet van de thesis}
\label{sec:opzet}
De rest van deze scriptie is als volgt opgebouwd:
\begin{itemize}
    \item \textbf{Hoofdstuk \ref{ch:analyse} (Analyse)} bespreekt de vereisten en het functionele ontwerp.
    \item \textbf{Hoofdstuk \ref{ch:technologie} (Technologie)} gaat dieper in op de gekozen tools en frameworks.
    \item \textbf{Hoofdstuk \ref{ch:implementatie} (Implementatie)} beschrijft de technische uitwerking en de code.
    \item \textbf{Hoofdstuk \ref{ch:conclusie} (Conclusie)} vat de resultaten samen en blikt vooruit naar mogelijke uitbreidingen.
\end{itemize}