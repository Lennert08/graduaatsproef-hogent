\chapter{Conclusie}
\label{ch:conclusie}

In deze scriptie hebben we een digitale handtekeningmodule ontwikkeld en geïntegreerd in het Recytix-platform. Het doel was om het contractproces, dat voorheen tijdrovend en handmatig verliep, te digitaliseren en te beveiligen.

\section{Evaluatie van de doelstellingen}
In de inleiding formuleerden we de initiële doelstelling om een werkend prototype te bouwen met SMS-verificatie en een juridische audit trail. 

We kunnen concluderen dat deze doelstellingen niet alleen behaald, maar **ruimschoots overtroffen** zijn. Omdat de ontwikkeling van de SMS-flow voorspoedig verliep, is er binnen de stageperiode ook een volledige integratie met **Itsme** gerealiseerd.

De behaalde resultaten zijn:
\begin{itemize}
    \item \textbf{Meerdere Authenticatiemethoden:} De gebruiker heeft nu de keuze uit twee flows:
    \begin{itemize}
        \item \textbf{SMS:} Via een 6-cijferige code.
        \item \textbf{Itsme:} Via de Itsme-identificatieflow (Login). Hierbij wordt het Rijksregisternummer opgehaald en gekoppeld aan de sessie als uniek bewijs van identiteit, waarna de gebruiker de ondertekening bevestigt.
    \end{itemize}
    \item \textbf{Integratie & Veiligheid:} De module draait succesvol binnen de Recytix-omgeving. Gevoelige data wordt veilig verwerkt en elke stap wordt onveranderlijk vastgelegd in de audit trail.
    \item \textbf{Overdraagbaarheid & Documentatie:} Om de continuïteit te waarborgen, is er uitgebreide technische documentatie opgeleverd in \textbf{Confluence}. Hierin staat beschreven hoe het component gebruikt moet worden en hoe toekomstige ontwikkelaars nieuwe providers (zoals WhatsApp) kunnen toevoegen, met een uitgewerkt code-voorbeeld.
\end{itemize}

De keuze voor de technologie-stack (React, .NET en PostgreSQL) is succesvol gebleken. De componenten sluiten naadloos aan op de bestaande architectuur van IRC Engineering.

\section{Toekomstperspectieven}
Hoewel het prototype nu al uitgebreider is dan gepland, zijn er nog logische vervolgstappen om de module verder te professionaliseren:

\begin{enumerate}
    \item \textbf{Meerdere ondertekenaars:} Momenteel ondersteunt de flow één ondertekenaar per document. In de toekomst kan dit uitgebreid worden zodat bijvoorbeeld zowel de klant als de leverancier hetzelfde document achtereenvolgens kunnen tekenen.
    \item \textbf{E-mailnotificaties:} Het automatisch versturen van een bevestigingsmail naar de ondertekenaar met het getekende document als bijlage zou de gebruikservaring compleet maken.
    \item \textbf{Uitbreiding providers:} Dankzij de generieke opzet en de handleiding in Confluence kan er eenvoudig ondersteuning worden toegevoegd voor andere kanalen, zoals WhatsApp for Business of buitenlandse ID-providers.
\end{enumerate}

\section{Slotwoord}
Dit project heeft aangetoond dat het mogelijk is om met moderne webtechnologieën een veilige en gebruiksvriendelijke ondertekenflow te integreren in een bestaand platform. Wat begon als een opdracht voor SMS-verificatie, is uitgegroeid tot een volwaardige module met Itsme-ondersteuning. De gerealiseerde oplossing vormt een solide basis voor verdere digitalisering binnen Recytix en biedt directe meerwaarde voor de eindgebruikers.