\chapter{Methodologie}
\label{ch:methodologie}

Voor de ontwikkeling van de applicatie is een aanpak gekozen die past binnen de werkwijze van IRC Engineering, maar rekening houdt met het individuele karakter van een graduaatsproef.

\section{Ontwikkelmethode}
Er is gekozen voor een **iteratieve aanpak** met een logische fase-opbouw:
\begin{itemize}
    \item \textbf{Fase 1 (Analyse \& Design):} Omdat er voor deze opdracht geen kant-en-klare tickets klaarlagen, zijn de functionele vereisten eerst vertaald naar een technisch ontwerp en een datamodel.
    \item \textbf{Fase 2 (Realisatie):} De ontwikkeling gebeurde stapsgewijs. Eerst is de volledige backend-logica (API en database) opgezet om de werking te garanderen. Vervolgens is de frontend hierop gebouwd. Deze volgorde zorgde ervoor dat de interface direct functioneel getest kon worden tegen een werkende backend.
    \item \textbf{Fase 3 (Testing \& Validatie):} Elke functionaliteit is na ontwikkeling direct gevalideerd op de testomgeving van Recytix.
\end{itemize}

\section{Projectbeheer (Jira)}
Binnen IRC Engineering wordt standaard gewerkt met **Jira** \cite{jira} voor het opvolgen van taken. In de dagelijkse werking hanteert het team een flexibele flow (Kanban-stijl) waarbij taken worden opgepakt naarmate ze binnenkomen, zonder strikte sprints.

Voor deze specifieke opdracht waren er echter geen vooraf gedefinieerde tickets. Ik heb daarom zelf de verantwoordelijkheid genomen om de grote opdracht (``bouw een tekenmodule'') op te splitsen in behapbare technische taken. Hoewel het project individueel was, is de methodiek van issue tracking gevolgd om structuur te behouden en de voortgang van backend, frontend en integraties (zoals Itsme) te bewaken.

\section{Gebruik van Generatieve AI}
Tijdens het ontwikkelproces is er bewust gebruikgemaakt van AI-assistenten, specifiek **Claude (Anthropic)** en **Google Gemini** \cite{claude, gemini}. Deze tools fungeerden als `digitale sparringpartner' en werden ingezet voor:

\begin{itemize}
    \item \textbf{Debugging:} Het analyseren van complexe foutmeldingen en stack traces in React en .NET.
    \item \textbf{Code-optimalisatie:} Het voorstellen van efficiëntere code-patronen (bijvoorbeeld het verbeteren van React Hooks).
    \item \textbf{Tekstverwerking:} Het controleren van spelling en het verfijnen van formuleringen in dit verslag en de Engelse samenvatting.
\end{itemize}

De gegenereerde output is steeds kritisch geëvalueerd, getest en waar nodig aangepast voordat deze in het project werd geïntegreerd.