\chapter{Methodologie}
\label{ch:methodologie}

Voor de ontwikkeling van de applicatie is een aanpak gekozen die past binnen de werkwijze van IRC Engineering, maar rekening houdt met het individuele karakter van een graduaatsproef.

\section{Ontwikkelmethode}
Er is gekozen voor een **iteratieve aanpak** met een logische fase-opbouw:
\begin{itemize}
    \item \textbf{Fase 1 (Analyse \& Design):} Omdat er voor deze opdracht geen kant-en-klare tickets klaarlagen, zijn de functionele vereisten eerst vertaald naar een technisch ontwerp en een datamodel.
    \item \textbf{Fase 2 (Realisatie):} De ontwikkeling gebeurde stapsgewijs. Eerst is de volledige backend-logica (API en database) opgezet om de werking te garanderen. Vervolgens is de frontend hierop gebouwd. Deze volgorde zorgde ervoor dat de interface direct functioneel getest kon worden tegen een werkende backend.
    \item \textbf{Fase 3 (Testing \& Validatie):} Elke functionaliteit is na ontwikkeling direct gevalideerd op de testomgeving van Recytix.
\end{itemize}

\section{Projectopvolging}
Binnen IRC Engineering wordt voor de dagelijkse werking gebruikgemaakt van **Jira** \cite{jira} om taken in een Kanban-flow op te volgen.

Aangezien deze graduaatsproef een individueel traject betrof met een duidelijk afgebakende scope, is er voor de uitvoering echter geen gebruikgemaakt van tickets. De focus lag op zelfstandige planning en uitvoering, waarbij de voortgang werd bewaakt door middel van directe feedbackmomenten met de stagementor en het aftoetsen van de gerealiseerde functionaliteiten aan de vereisten uit de analysefase.

% --- HARD PAGE BREAK OM DUBBELE TITEL TE VOORKOMEN ---
\clearpage
% ----------------------------------------------------

\section{Gebruik van Generatieve AI}
Tijdens het ontwikkelproces is er bewust gebruikgemaakt van AI-assistenten, specifiek **Claude (Anthropic)** en **Google Gemini** \cite{claude, gemini}. Deze tools fungeerden als `digitale sparringpartner' en werden ingezet voor:

\begin{itemize}
    \item \textbf{Debugging:} Het analyseren van complexe foutmeldingen en stack traces in React en .NET.
    \item \textbf{Code-optimalisatie:} Het voorstellen van efficiëntere code-patronen (bijvoorbeeld het verbeteren van React Hooks).
    \item \textbf{Tekstverwerking:} Het controleren van spelling en het verfijnen van formuleringen in dit verslag en de Engelse samenvatting.
\end{itemize}

De gegenereerde output is steeds kritisch geëvalueerd, getest en waar nodig aangepast voordat deze in het project werd geïntegreerd.