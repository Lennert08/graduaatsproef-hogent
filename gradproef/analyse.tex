\chapter{Analyse}
\label{ch:analyse}

Voordat de ontwikkeling start, is het cruciaal om de vereisten van de applicatie helder in kaart te brengen. In dit hoofdstuk beschrijven we de functionele en niet-functionele vereisten van de digitale ondertekenmodule binnen Recytix, evenals de gewenste flow voor de eindgebruiker.

\section{Functionele Vereisten}
De applicatie moet de volgende kernfunctionaliteiten bieden:

\begin{itemize}
    \item \textbf{Documentweergave:} De gebruiker moet het te ondertekenen document (PDF-formaat) duidelijk kunnen inzien in de browser, zonder externe software te hoeven downloaden.
    \item \textbf{Authenticatie (SMS \& Itsme):} Om de identiteit van de ondertekenaar te verifiëren, moet er gebruikgemaakt worden van Two-Factor Authentication (2FA). Dit kan via een SMS-code of via de Itsme-app. De gebruiker kan het document pas ondertekenen na een succesvolle verificatie.
    \item \textbf{Audit Trail:} Elke stap in het proces moet worden gelogd. Dit omvat: het openen van het document, de keuze van authenticatie, en de succesvolle verificatie. Dit logboek dient als bewijs voor de juridische geldigheid.
    \item \textbf{Statusbeheer:} Het systeem moet bijhouden of een document 'In behandeling', 'Getekend' of 'Geweigerd' is.
\end{itemize}

\section{Niet-functionele Vereisten}
Naast de functies zijn er ook kwaliteitseisen:
\begin{itemize}
    \item \textbf{Veiligheid:} De communicatie tussen client en server moet versleuteld zijn. Persoonsgegevens (zoals telefoonnummers) moeten veilig worden verwerkt en opgeslagen.
    \item \textbf{Integratie:} De module moet visueel en technisch naadloos integreren in de bestaande Recytix-omgeving (huisstijl, login-sessie).
    \item \textbf{Responsiviteit:} De interface moet werken op zowel desktop als mobiele apparaten, aangezien gebruikers vaak ondertekenen via hun smartphone.
\end{itemize}

% --- HARD PAGE BREAK OM LAYOUT TE FIXEN ---
\clearpage
% ------------------------------------------

\section{Gewenste Flow (User Journey)}
\label{sec:user_journey}

Het proces voor de eindgebruiker is ontworpen als een intuïtieve multi-step flow, geïntegreerd in de bestaande Recytix-omgeving. De stappen zijn als volgt gedefinieerd:

\begin{enumerate}
    \item \textbf{Inloggen en Notificatie:} 
    Zodra de gebruiker inlogt op het portaal, controleert het systeem of er openstaande documenten zijn. Indien dit het geval is, verschijnt er rechtsonder een \textit{toast}-melding (bijvoorbeeld: "U heeft nog 5 te ondertekenen contracten").
    
    \item \textbf{Navigatie naar Overzicht:} 
    Door op de melding te klikken, wordt de gebruiker direct naar de contracten-grid geleid. In dit overzicht kan de gebruiker de status van alle documenten zien en de lijst sorteren of filteren op 'Nog te ondertekenen'.
    
    \item \textbf{Inzage:} 
    De gebruiker selecteert een document en krijgt een voorvertoning (PDF) te zien binnen de interface.
    
    \item \textbf{Start Ondertekenproces:} 
    Na het klikken op de knop "Ondertekenen", opent een \textit{multi-step wizard}. De gebruiker krijgt hier de keuze tussen twee authenticatiemethoden:
    \begin{itemize}
        \item \textbf{SMS-verificatie:} Er wordt een code naar het gekoppelde mobiele nummer gestuurd.
        \item \textbf{Itsme:} Authenticatie verloopt via de Itsme-app (indien geconfigureerd).
    \end{itemize}
    
    \item \textbf{Verificatie:} 
    De gebruiker voert de ontvangen SMS-code in of bevestigt zijn identiteit via de Itsme-app.
    
    \item \textbf{Afronding:} 
    Bij een succesvolle verificatie wordt het document in de database gemarkeerd als 'Getekend'. De audit trail wordt weggeschreven en de gebruiker krijgt een succesmelding te zien, waarna de status in de grid direct wordt bijgewerkt.
\end{enumerate}

\section{Datamodel}
Om dit proces te ondersteunen, moeten in de database minimaal de volgende entiteiten worden opgeslagen:
\begin{itemize}
    \item \textbf{Document:} Metadata over het bestand (bestandsnaam, locatie, eigenaar).
    \item \textbf{SignRequest:} De koppeling tussen een document en een ondertekenaar, inclusief de huidige status.
    \item \textbf{AuditLog:} Een chronologische lijst van events (tijdstempel, actie, IP-adres, gebruiker).
\end{itemize}