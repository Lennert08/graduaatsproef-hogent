\documentclass[dutch,dit,thesis,oneside]{hogentreport}

\usepackage{lipsum}
\usepackage[chapter,outputdir=../output]{minted}
\usemintedstyle{solarized-light}

\setminted{
    autogobble,
    frame=lines,
    breaklines,
    linenos,
    tabsize=4
}

\renewcommand\listoflistingscaption{\IfLanguageName{dutch}{Lijst van codefragmenten}{List of listings}}
\renewcommand\listingscaption{\IfLanguageName{dutch}{Codefragment}{Listing}}
\renewcommand*\listoflistings{\cleardoublepage\phantomsection\addcontentsline{toc}{chapter}{\listoflistingscaption}\listof{listing}{\listoflistingscaption}}

%%---------- JOUW GEGEVENS -------------------------------------------------
\title{Ontwikkeling van een generieke digitale ondertekenmodule met SMS- en Itsme-integratie}
\author{Lennert Van Daele}
\supervisor{Tommy Uytterhaegen}
\cosupervisor{Kris De Ridder (IRC Engineering)}
\academicyear{2025-2026}
\examperiod{1}
\degreesought{Graduaat in het Programmeren}
\partialthesis{false}

\addbibresource{gradproef.bib}
\renewcommand{\cleardoublepage}{\clearpage}
\renewcommand{\arraystretch}{1.2}

\begin{document}

\frontmatter
\hypersetup{pageanchor=false}
\maketitle
\hypersetup{pageanchor=true}

% --- VOLGORDE AANGEPAST AAN SCHOOLREGELS ---
% 1. Eerst de Nederlandse samenvatting
\chapter{Samenvatting}
\label{ch:samenvatting}

Deze graduaatsproef beschrijft de ontwikkeling en integratie van een digitale handtekeningmodule binnen het cloudplatform Recytix. Voorheen ontbrak een geïntegreerde oplossing om documenten, zoals contracten en mandaten, digitaal te laten ondertekenen. Dit leidde tot inefficiënte processen en administratieve rompslomp.

Het doel van dit project was het realiseren van een werkend prototype waarmee gebruikers documenten juridisch geldig kunnen ondertekenen via SMS-verificatie. Tijdens het traject is dit doel overtroffen door ook authenticatie via **Itsme** volledig te integreren. De focus lag hierbij op gebruiksvriendelijkheid, veiligheid en schaalbaarheid.

Voor de realisatie is gebruikgemaakt van een moderne technologie-stack: een frontend in \textbf{React} (TypeScript) en een backend in \textbf{C\# (.NET)}. De communicatie verloopt via een RESTful API en data wordt veilig opgeslagen in een \textbf{PostgreSQL}-database.

De kernfunctionaliteit omvat het inzien van PDF-documenten en een keuze-flow tussen SMS- of Itsme-verificatie. Elke stap wordt nauwkeurig vastgelegd in een audit trail. Daarnaast is er uitgebreide technische documentatie opgeleverd in Confluence om toekomstige uitbreidingen (zoals WhatsApp-integratie) te faciliteren.

Het resultaat is een functionele, veilige module die naadloos integreert met het Recytix-platform. Hiermee wordt de doorlooptijd van administratieve processen aanzienlijk verkort en is een solide basis gelegd voor verdere digitalisering.

\vspace{1.5cm}
\noindent \textbf{Link naar GitHub repository (Verslag):} \\
\url{https://github.com/Lennert08/graduaatsproef-hogent}

% 2. Dan de Engelse abstract
\chapter{Summary}
\label{ch:summary}

This associate thesis describes the development and integration of a digital signature module within the cloud platform Recytix. Previously, an integrated solution for digitally signing documents, such as contracts and mandates, was missing. This resulted in inefficient processes and administrative burdens.

The initial goal was to create a prototype for signing documents via SMS verification. This goal was exceeded during the internship by also fully integrating **Itsme** authentication. The focus remained on usability, security, and scalability.

A modern technology stack was chosen: a frontend in \textbf{React} (TypeScript) and a backend in \textbf{C\# (.NET)}. Communication is handled via a RESTful API, and data is securely stored in a \textbf{PostgreSQL} database.

The core functionality includes viewing PDF documents and a choice flow between SMS or Itsme verification. Every step is accurately recorded in an audit trail. Additionally, comprehensive technical documentation was delivered in Confluence to facilitate future expansions (such as WhatsApp integration).

The result is a functional, secure module that seamlessly integrates with the Recytix platform. This significantly reduces the turnaround time of administrative processes and lays a solid foundation for further digitization.

\vspace{1.5cm}
\noindent \textbf{Link to GitHub repository (Report):} \\
\url{https://github.com/Lennert08/graduaatsproef-hogent}

% 3. Dan pas het voorwoord
\chapter*{Voorwoord}

Voor u ligt de graduaatsproef "Ontwikkeling van een generieke digitale ondertekenmodule met SMS- en Itsme-integratie". Dit werk vormt de afronding van mijn opleiding Graduaat Programmeren aan HOGENT. Het project werd uitgevoerd tijdens mijn eindstage bij IRC Engineering in Dendermonde, waar ik de kans kreeg om mee te werken aan de applicatie Recytix.

Het realiseren van dit project was een leerrijke ervaring die niet mogelijk was geweest zonder de steun en begeleiding van verschillende personen.

Allereerst wil ik mijn mentor, \textbf{Kris De Ridder} (Senior Project Engineer), bedanken. Kris, bedankt voor het vertrouwen dat je me gaf om meteen aan de slag te gaan met de core-technologieën van het bedrijf. Je deur stond altijd open voor vragen, en je feedback tijdens de code-reviews heeft me enorm geholpen om een betere developer te worden. Ook bedankt voor de ruimte die ik kreeg om zelfstandig de Itsme-integratie te onderzoeken en uit te werken.

Daarnaast wil ik mijn promotor vanuit HOGENT, \textbf{Tommy Uytterhaegen}, bedanken voor de opvolging en de constructieve feedbackmomenten tijdens dit academiejaar.

Een speciaal woord van dank gaat uit naar het volledige team van \textbf{IRC Engineering}. Vanaf dag één werd ik warm onthaald en behandeld als een volwaardig lid van het team. De aangename werksfeer en de behulpzaamheid van de collega's maakten dat ik elke dag met plezier naar kantoor kwam.

Tot slot wil ik mijn familie en vrienden bedanken voor hun steun en interesse tijdens mijn studie en deze stageperiode.

\vspace{1cm}
\noindent Lennert Van Daele \\
Dendermonde, \today
% -------------------------------------------

\tableofcontents
\listoffigures

\mainmatter
\chapter{Inleiding}
\label{ch:inleiding}

In dit inleidende hoofdstuk schetsen we de context van het project, definiëren we de probleemstelling en formuleren we de doelstellingen. Daarnaast bakenen we de scope van het project af en geven we een overzicht van de structuur van deze scriptie.

\section{Probleemstelling}
\label{sec:probleemstelling}
Het platform Recytix heeft momenteel geen geïntegreerd systeem om documenten digitaal te laten ondertekenen. Voor cruciale documenten, zoals bijvoorbeeld verkoopovereenkomsten of mandaten, is een efficiënt en juridisch sluitend proces essentieel. Het ontbreken van deze functionaliteit zorgt voor administratieve vertragingen en een minder vlotte klantervaring.

Dit project vult die nood in door een specifieke module te ontwikkelen die naadloos aansluit op het bestaande ecosysteem.

\section{Doelstelling}
\label{sec:doelstelling}
Het hoofddoel van deze bachelorproef is het bouwen van een werkend prototype van een digitale handtekeningmodule binnen Recytix. 

Concreet zijn de subdoelen:
\begin{itemize}
    \item Het creëren van een algemeen toepasbare oplossing waarmee gebruikers documenten kunnen inzien.
    \item Het implementeren van een ondertekenflow via een SMS-code (Two-Factor Authentication).
    \item Het garanderen van juridische geldigheid door middel van een sluitende audit trail (logboek van acties).
\end{itemize}

\section{Onderzoeksvragen}
\label{sec:onderzoeksvragen}
Op basis van de probleemstelling en doelstelling formuleren we de volgende centrale onderzoeksvraag:
\begin{quote}
    \textit{Hoe kan een digitale handtekeningmodule met SMS-integratie en audit trail geïntegreerd worden in het Recytix-platform om het contractproces te digitaliseren?}
\end{quote}

\section{Methodologie}
\label{sec:methodologie}
Voor de ontwikkeling van deze applicatie wordt gebruikgemaakt van de volgende technologieën:
\begin{itemize}
    \item \textbf{Frontend:} React voor de gebruikersinterface.
    \item \textbf{Backend:} C\# (.NET) voor de API en logica.
    \item \textbf{Database:} PostgreSQL voor dataopslag en de audit trail.
\end{itemize}

De ontwikkeling verloopt iteratief, waarbij eerst de analyse en architectuur worden opgezet, gevolgd door de implementatie van de backend en vervolgens de frontend.

\section{Opzet van de thesis}
\label{sec:opzet}
De rest van deze scriptie is als volgt opgebouwd:
\begin{itemize}
    \item \textbf{Hoofdstuk \ref{ch:analyse} (Analyse)} bespreekt de vereisten en het functionele ontwerp.
    \item \textbf{Hoofdstuk \ref{ch:technologie} (Technologie)} gaat dieper in op de gekozen tools en frameworks.
    \item \textbf{Hoofdstuk \ref{ch:implementatie} (Implementatie)} beschrijft de technische uitwerking en de code.
    \item \textbf{Hoofdstuk \ref{ch:conclusie} (Conclusie)} vat de resultaten samen en blikt vooruit naar mogelijke uitbreidingen.
\end{itemize}
\chapter{Analyse}
\label{ch:analyse}

Voordat de ontwikkeling start, is het cruciaal om de vereisten van de applicatie helder in kaart te brengen. In dit hoofdstuk beschrijven we de functionele en niet-functionele vereisten van de digitale ondertekenmodule binnen Recytix, evenals de gewenste flow voor de eindgebruiker.

\section{Functionele Vereisten}
De applicatie moet de volgende kernfunctionaliteiten bieden:

\begin{itemize}
    \item \textbf{Documentweergave:} De gebruiker moet het te ondertekenen document (PDF-formaat) duidelijk kunnen inzien in de browser, zonder externe software te hoeven downloaden.
    \item \textbf{Authenticatie (SMS \& Itsme):} Om de identiteit van de ondertekenaar te verifiëren, moet er gebruikgemaakt worden van Two-Factor Authentication (2FA). Dit kan via een SMS-code of via de Itsme-app. De gebruiker kan het document pas ondertekenen na een succesvolle verificatie.
    \item \textbf{Audit Trail:} Elke stap in het proces moet worden gelogd. Dit omvat: het openen van het document, de keuze van authenticatie, en de succesvolle verificatie. Dit logboek dient als bewijs voor de juridische geldigheid.
    \item \textbf{Statusbeheer:} Het systeem moet bijhouden of een document 'In behandeling', 'Getekend' of 'Geweigerd' is.
\end{itemize}

\section{Niet-functionele Vereisten}
Naast de functies zijn er ook kwaliteitseisen:
\begin{itemize}
    \item \textbf{Veiligheid:} De communicatie tussen client en server moet versleuteld zijn. Persoonsgegevens (zoals telefoonnummers) moeten veilig worden verwerkt en opgeslagen.
    \item \textbf{Integratie:} De module moet visueel en technisch naadloos integreren in de bestaande Recytix-omgeving (huisstijl, login-sessie).
    \item \textbf{Responsiviteit:} De interface moet werken op zowel desktop als mobiele apparaten, aangezien gebruikers vaak ondertekenen via hun smartphone.
\end{itemize}

% --- HARD PAGE BREAK OM LAYOUT TE FIXEN ---
\clearpage
% ------------------------------------------

\section{Gewenste Flow (User Journey)}
\label{sec:user_journey}

Het proces voor de eindgebruiker is ontworpen als een intuïtieve multi-step flow, geïntegreerd in de bestaande Recytix-omgeving. De stappen zijn als volgt gedefinieerd:

\begin{enumerate}
    \item \textbf{Inloggen en Notificatie:} 
    Zodra de gebruiker inlogt op het portaal, controleert het systeem of er openstaande documenten zijn. Indien dit het geval is, verschijnt er rechtsonder een \textit{toast}-melding (bijvoorbeeld: "U heeft nog 5 te ondertekenen contracten").
    
    \item \textbf{Navigatie naar Overzicht:} 
    Door op de melding te klikken, wordt de gebruiker direct naar de contracten-grid geleid. In dit overzicht kan de gebruiker de status van alle documenten zien en de lijst sorteren of filteren op 'Nog te ondertekenen'.
    
    \item \textbf{Inzage:} 
    De gebruiker selecteert een document en krijgt een voorvertoning (PDF) te zien binnen de interface.
    
    \item \textbf{Start Ondertekenproces:} 
    Na het klikken op de knop "Ondertekenen", opent een \textit{multi-step wizard}. De gebruiker krijgt hier de keuze tussen twee authenticatiemethoden:
    \begin{itemize}
        \item \textbf{SMS-verificatie:} Er wordt een code naar het gekoppelde mobiele nummer gestuurd.
        \item \textbf{Itsme:} Authenticatie verloopt via de Itsme-app (indien geconfigureerd).
    \end{itemize}
    
    \item \textbf{Verificatie:} 
    De gebruiker voert de ontvangen SMS-code in of bevestigt zijn identiteit via de Itsme-app.
    
    \item \textbf{Afronding:} 
    Bij een succesvolle verificatie wordt het document in de database gemarkeerd als 'Getekend'. De audit trail wordt weggeschreven en de gebruiker krijgt een succesmelding te zien, waarna de status in de grid direct wordt bijgewerkt.
\end{enumerate}

\section{Datamodel}
Om dit proces te ondersteunen, moeten in de database minimaal de volgende entiteiten worden opgeslagen:
\begin{itemize}
    \item \textbf{Document:} Metadata over het bestand (bestandsnaam, locatie, eigenaar).
    \item \textbf{SignRequest:} De koppeling tussen een document en een ondertekenaar, inclusief de huidige status.
    \item \textbf{AuditLog:} Een chronologische lijst van events (tijdstempel, actie, IP-adres, gebruiker).
\end{itemize}
\chapter{Technologie}
\label{ch:technologie}

In dit hoofdstuk lichten we de technologie-stack toe die gekozen is voor de realisatie van de applicatie. De keuzes zijn gebaseerd op de standaarden binnen IRC Engineering en de specifieke vereisten van dit project.

\section{Frontend: React}
Voor de gebruikersinterface maken we gebruik van \textbf{React}. React is een populaire JavaScript-bibliotheek voor het bouwen van user interfaces, ontwikkeld door Meta.

De belangrijkste redenen om voor React te kiezen zijn:
\begin{itemize}
    \item \textbf{Component-based architecture:} Dit maakt het mogelijk om herbruikbare UI-elementen te bouwen, wat de consistentie in de applicatie bevordert.
    \item \textbf{Virtual DOM:} Dit zorgt voor efficiënte updates van de pagina, wat essentieel is voor een vlotte gebruikerservaring tijdens het ondertekenproces.
    \item \textbf{Ecosysteem:} React heeft een enorme community en veel beschikbare libraries, wat de ontwikkeling versnelt.
\end{itemize}

\section{Backend: C\# en .NET}
De logica van de applicatie en de API worden ontwikkeld in \textbf{C\#} met behulp van het \textbf{.NET framework}. Dit is de standaard backend-technologie binnen IRC Engineering.

De voordelen van .NET in deze context zijn:
\begin{itemize}
    \item \textbf{Strong typing:} C\# is een sterk getypeerde taal, wat helpt om fouten vroegtijdig tijdens de ontwikkeling op te sporen.
    \item \textbf{Security:} .NET biedt ingebouwde functionaliteiten voor authenticatie en autorisatie, wat cruciaal is voor het veilig verwerken van handtekeningen.
    \item \textbf{Performance:} De moderne versies van .NET zijn sterk geoptimaliseerd en bieden hoge prestaties voor web-API’s.
\end{itemize} 
% HIER ZAT DE FOUT: De itemize moet afgesloten zijn VOOR de volgende sectie begint.

\section{Database: PostgreSQL}
Voor de opslag van gegevens, inclusief de document-metadata en de audit trail, gebruiken we \textbf{PostgreSQL}.

PostgreSQL staat bekend om zijn betrouwbaarheid en robuustheid. Voor dit project is vooral de ondersteuning voor complexe queries en transacties belangrijk, zodat we kunnen garanderen dat een handtekening altijd correct en onveranderlijk wordt opgeslagen in de audit trail.

\section{Hosting en Ontwikkelomgeving}
De applicatie wordt gehost binnen de infrastructuur van Recytix, die volledig draait op **Linux Debian** servers. De handtekeningmodule functioneert als een microservice binnen dit ecosysteem, waardoor het direct gebruik kan maken van de bestaande gebruikersaccounts en rechtenstructuren.

Een specifieke technische uitdaging tijdens de ontwikkeling was de integratie met **Itsme**. De authenticatie-flow van Itsme vereist strikte veiligheidsprotocollen en functioneert niet op een lokale ontwikkelomgeving (`localhost`). Hierdoor was het niet mogelijk om de volledige flow lokaal te testen. Voor elke test-iteratie van de Itsme-integratie moest de applicatie effectief gedeployed worden naar een online testomgeving op de Linux-servers, wat zorgde voor een complexere ontwikkelcyclus.
\chapter{Implementatie}
\label{ch:implementatie}

In dit hoofdstuk beschrijven we de technische realisatie van de applicatie. We splitsen dit op in de backend-implementatie (API en database) en de frontend-implementatie. Vanwege de vertrouwelijkheid van de broncode worden hier geen letterlijke codefragmenten getoond, maar focussen we op de architectuur, de datastructuren en de logische flows.

\section{Backend Implementatie (.NET)}
De backend fungeert als de centrale spil tussen de database, de frontend en de SMS-provider. De focus lag hier op veiligheid en robuustheid.

\subsection{Datamodel}
Om het ondertekenproces te faciliteren, is het datamodel uitgebreid met enkele nieuwe entiteiten. De belangrijkste entiteit is de \textbf{SignRequest}. Deze koppelt een specifiek document aan een gebruiker en houdt de status van het proces bij.

Belangrijke eigenschappen die we hierbij opslaan zijn:
\begin{itemize}
    \item De unieke identificatie van het document en de gebruiker.
    \item Het telefoonnummer waarnaar de verificatiecode is verzonden.
    \item De gegenereerde verificatiecode (versleuteld opgeslagen).
    \item De status van de handtekening (bijv. \textit{Pending}, \textit{Signed}, \textit{Expired}).
    \item Tijdstempels voor audit-doeleinden (aanmaakdatum, tekenmoment).
\end{itemize}

\subsection{API Controllers}
De communicatie verloopt via een RESTful API. De \texttt{SigningController} bevat de endpoints die nodig zijn voor de flow:
\begin{enumerate}
    \item \textbf{POST /request:} Start het proces. Deze methode genereert een cryptografisch veilige random code en slaat deze op in de database. Vervolgens wordt de SMS-service aangeroepen.
    \item \textbf{POST /verify:} Verifieert de ingevoerde code. Hier wordt gecontroleerd of de code overeenkomt met de opgeslagen hash en of de code nog niet verlopen is. Bij succes wordt de status van het document geüpdatet.
\end{enumerate}

\section{Frontend Implementatie (React)}
De frontend is ontwikkeld in React met TypeScript om type-veiligheid te garanderen. De implementatie richtte zich vooral op een gebruiksvriendelijke ervaring (UX).

\subsection{User Interface Componenten}
Er is een modale component ontwikkeld die over de document-viewer heen ligt. Deze component heeft verschillende statussen:
\begin{itemize}
    \item \textbf{Loading:} Terwijl de SMS verstuurd wordt.
    \item \textbf{Input:} Waar de gebruiker de 6-cijferige code kan invoeren.
    \item \textbf{Error:} Indien de code onjuist is of er een serverfout optreedt.
    \item \textbf{Success:} Een bevestiging dat het document getekend is.
\end{itemize}

\subsection{Communicatie met de Backend}
De frontend maakt gebruik van asynchrone HTTP-calls om met de backend te praten. Er is foutafhandeling ingebouwd om de gebruiker te informeren als de SMS niet verstuurd kan worden of als de sessie verlopen is.

\section{Integratie in Recytix}
De grootste uitdaging was de naadloze integratie in het bestaande Recytix-platform. Omdat Recytix al over een authenticatiesysteem beschikt, hebben we de module zo gebouwd dat deze de huidige gebruikerscontext (via JWT-tokens) hergebruikt. Hierdoor hoeft de gebruiker niet apart in te loggen om te tekenen; het systeem weet direct wie de ingelogde persoon is en welk telefoonnummer daarbij hoort.
\chapter{Conclusie}
\label{ch:conclusie}

In deze scriptie hebben we een digitale handtekeningmodule ontwikkeld en geïntegreerd in het Recytix-platform. Het doel was om het contractproces, dat voorheen tijdrovend en handmatig verliep, te digitaliseren en te beveiligen.

\section{Evaluatie van de doelstellingen}
In de inleiding formuleerden we de doelstelling om een werkend prototype te bouwen met SMS-verificatie en een juridische audit trail. 

We kunnen concluderen dat deze doelstellingen behaald zijn:
\begin{itemize}
    \item \textbf{Integratie:} De module draait succesvol binnen de Recytix-omgeving. Gebruikers kunnen documenten inzien zonder de applicatie te verlaten.
    \item \textbf{Veiligheid:} De Two-Factor Authentication (2FA) via SMS werkt correct en zorgt voor de noodzakelijke identiteitsverificatie.
    \item \textbf{Traceerbaarheid:} Elke actie wordt vastgelegd in de database (audit trail), waardoor er achteraf altijd bewijs is van wie wanneer getekend heeft.
\end{itemize}

De keuze voor de technologie-stack (React, .NET en PostgreSQL) is succesvol gebleken. De componenten sluiten naadloos aan op de bestaande architectuur van IRC Engineering, wat de onderhoudbaarheid ten goede komt.

\section{Toekomstperspectieven}
Hoewel het huidige prototype functioneel is, zijn er nog mogelijkheden tot uitbreiding om de module nog krachtiger te maken:

\begin{enumerate}
    \item \textbf{Itsme-integratie:} De huidige opzet is generiek gebouwd. Een logische vervolgstap is het toevoegen van Itsme als alternatieve verificatiemethode naast SMS, voor een nog hogere betrouwbaarheidsgraad.
    \item \textbf{Meerdere ondertekenaars:} Momenteel ondersteunt de flow één ondertekenaar per document. In de toekomst kan dit uitgebreid worden zodat bijvoorbeeld zowel de klant als de leverancier hetzelfde document achtereenvolgens kunnen tekenen.
    \item \textbf{E-mailnotificaties:} Het automatisch versturen van een bevestigingsmail met het getekende document als bijlage zou de gebruikservaring compleet maken.
\end{enumerate}

\section{Slotwoord}
Dit project heeft aangetoond dat het mogelijk is om met moderne webtechnologieën een veilige en gebruiksvriendelijke ondertekenflow te integreren in een bestaand platform. De gerealiseerde module vormt een solide basis voor verdere digitalisering binnen Recytix en biedt directe meerwaarde voor de eindgebruikers.

\backmatter
\setlength\bibitemsep{2pt}
\printbibliography[heading=bibintoc]

\end{document}