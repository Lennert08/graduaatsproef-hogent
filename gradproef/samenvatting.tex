\chapter{Samenvatting}
\label{ch:samenvatting}

Deze graduaatsproef beschrijft de ontwikkeling en integratie van een digitale handtekeningmodule binnen het cloudplatform Recytix. Voorheen ontbrak een geïntegreerde oplossing om documenten, zoals contracten en mandaten, digitaal te laten ondertekenen. Dit leidde tot inefficiënte processen en administratieve rompslomp.

Het doel van dit project was het realiseren van een werkend prototype waarmee gebruikers documenten juridisch geldig kunnen ondertekenen via SMS-verificatie. Tijdens het traject is dit doel overtroffen door ook authenticatie via **Itsme** volledig te integreren. De focus lag hierbij op gebruiksvriendelijkheid, veiligheid en schaalbaarheid.

Voor de realisatie is gebruikgemaakt van een moderne technologie-stack: een frontend in \textbf{React} (TypeScript) en een backend in \textbf{C\# (.NET)}. De communicatie verloopt via een RESTful API en data wordt veilig opgeslagen in een \textbf{PostgreSQL}-database.

De kernfunctionaliteit omvat het inzien van PDF-documenten en een keuze-flow tussen SMS- of Itsme-verificatie. Elke stap wordt nauwkeurig vastgelegd in een audit trail. Daarnaast is er uitgebreide technische documentatie opgeleverd in Confluence om toekomstige uitbreidingen (zoals WhatsApp-integratie) te faciliteren.

Het resultaat is een functionele, veilige module die naadloos integreert met het Recytix-platform. Hiermee wordt de doorlooptijd van administratieve processen aanzienlijk verkort en is een solide basis gelegd voor verdere digitalisering.

\vspace{1.5cm}
\noindent \textbf{Link naar GitHub repository (Verslag):} \\
\url{https://github.com/Lennert08/graduaatsproef-hogent}