\chapter{Samenvatting}
\label{ch:samenvatting}

Deze graduaatsproef beschrijft de ontwikkeling en integratie van een digitale handtekeningmodule binnen het cloudplatform Recytix. Voorheen ontbrak een geïntegreerde oplossing om documenten, zoals contracten en mandaten, digitaal te laten ondertekenen. Dit leidde tot inefficiënte processen en administratieve rompslomp.

Het doel van dit project was het realiseren van een werkend prototype waarmee gebruikers documenten juridisch geldig kunnen ondertekenen zonder de applicatie te verlaten. De focus lag hierbij op gebruiksvriendelijkheid en veiligheid.

Voor de realisatie is gebruikgemaakt van een moderne technologie-stack die aansluit bij de standaarden van IRC Engineering. De frontend is ontwikkeld in \textbf{React} met TypeScript, wat zorgt voor een responsieve gebruikersinterface. De backend is gerealiseerd in \textbf{C\# (.NET)}, waarbij een RESTful API de communicatie verzorgt. Voor de dataopslag en de audit trail wordt gebruikgemaakt van een \textbf{PostgreSQL}-database.

De kernfunctionaliteit van de module omvat het inzien van PDF-documenten en het ondertekenen via een Two-Factor Authentication (2FA) mechanisme per SMS. Elke stap in het proces wordt nauwkeurig vastgelegd in een audit trail, wat essentieel is voor de bewijslast.

Het resultaat is een functionele en veilige module die naadloos integreert met de bestaande gebruikersauthenticatie van Recytix. Hiermee wordt de doorlooptijd van administratieve processen aanzienlijk verkort en wordt een solide basis gelegd voor verdere uitbreidingen, zoals integratie met Itsme of ondersteuning voor meerdere ondertekenaars.

\vspace{1.5cm}
\noindent \textbf{Link naar GitHub repository (Verslag):} \\
\url{https://github.com/Lennert08/graduaatsproef-hogent}