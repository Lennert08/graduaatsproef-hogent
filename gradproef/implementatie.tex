\chapter{Implementatie}
\label{ch:implementatie}

In dit hoofdstuk bespreken we de effectieve realisatie van de applicatie. We splitsen dit op in de backend-logica (.NET), de frontend-integratie (React) en de specifieke beveiligingskeuzes die zijn gemaakt.

\section{Backend Architectuur (.NET)}
De backend fungeert als de motor van de applicatie en beheert de communicatie met de database en externe services.

\subsection{Datamodel en Sessiebeheer}
De kern van de module draait om de \textbf{SessionId}. Dit is de unieke identifier (Primary Key) die wordt aangemaakt zodra een gebruiker het ondertekenproces start. 

In tegenstelling tot complexe state-machines, is er bewust gekozen voor een binaire statusbenadering voor de documenten: een document is ofwel \textbf{Niet Ondertekend}, ofwel \textbf{Ondertekend}. Er is geen tussenliggende 'Pending' status in de database; de voortgang van de gebruiker wordt puur bijgehouden via de sessie en de audit trail.

\subsection{SMS Integratie (smstools)}
Voor het versturen van de verificatiecodes maken we geen gebruik van een eigen mailserver of GSM-gateway, maar integreren we met de externe API van \textbf{smstools}.

Wanneer de gebruiker kiest voor SMS-verificatie:
\begin{enumerate}
    \item De backend genereert een willekeurige 6-cijferige code.
    \item Deze code wordt gehasht opgeslagen in de database (zie sectie \ref{sec:security}).
    \item Er wordt een HTTP POST-request gestuurd naar de API van smstools met het mobiele nummer van de gebruiker en de code.
    \item Smstools zorgt voor de daadwerkelijke aflevering van het bericht.
\end{enumerate}

\section{Frontend Implementatie (React)}
De frontend is gebouwd als een herbruikbare component die naadloos in de Recytix-portal laadt.

\subsection{State Management}
Om te weten waar de gebruiker zich bevindt in het proces (bijv. "Code ingevuld" of "Nog aan het lezen"), wordt de unieke \textbf{SessionId} opgeslagen in de \textbf{sessionStorage} van de browser.

Er is specifiek gekozen voor `sessionStorage` in plaats van `localStorage` of cookies, omdat de sessie hierdoor automatisch wordt gewist zodra de gebruiker het tabblad of de browser sluit. Dit verhoogt de veiligheid: een volgende gebruiker op dezelfde PC kan niet per ongeluk verdergaan in een oude sessie. Er is in deze fase geen gebruikgemaakt van complexe JWT-implementaties, aangezien de sessie-ID voldoende context biedt binnen de beveiligde Recytix-omgeving.

\subsection{Meertaligheid (i18n)}
Aangezien Recytix klanten heeft in heel België en daarbuiten, is de volledige module **multi-language** opgezet. Alle teksten, knoppen en foutmeldingen zijn beschikbaar in het **Nederlands (NL)**, **Frans (FR)** en **Duits (DE)**. De taal wordt automatisch ingesteld op basis van de voorkeur van de ingelogde gebruiker.

\section{Beveiliging: Hashing vs. Encryptie}
\label{sec:security}
Bij het opslaan van gevoelige data, zoals de verificatiecodes in de audit trail, is er bewust gekozen voor **hashing** in plaats van encryptie.

\begin{itemize}
    \item **Waarom Hashing?** Hashing is eenrichtingsverkeer. We moeten de code later kunnen verifiëren (door de invoer van de gebruiker te hashen en te vergelijken), maar we hoeven de originele code nooit meer terug te kunnen lezen uit de database.
    \item **Voordeel:** Mocht de database ooit gecompromitteerd worden, dan zijn de verificatiecodes onleesbaar en niet terug te rekenen naar de originele waarden.
\end{itemize}

De audit trail fungeert hierdoor als een veilig logboek dat bewijst \textit{dat} de juiste code is ingevoerd, zonder de code zelf bloot te geven.