\section{Context en probleemstelling}
Het platform Energytix heeft momenteel geen geïntegreerd systeem om documenten digitaal te laten ondertekenen. Voor cruciale documenten, zoals bijvoorbeeld verkoopovereenkomsten of mandaten, is een efficiënt en juridisch sluitend proces essentieel. Dit project vult die nood in door een specifieke module te ontwikkelen.

\section{Doelstelling}
Ik bouw een werkend prototype van een digitale handtekeningmodule die naadloos kan worden geïntegreerd in Energytix. Het doel is om een algemeen toepasbare oplossing te creëren waarmee gebruikers documenten digitaal kunnen ondertekenen via een sms-code, inclusief een sluitende audit trail voor juridische geldigheid.

\section{Technologieën}
\begin{itemize}
    \item Frontend: React
    \item Backend: C\# (.NET)
    \item Database: PostgreSQL
\end{itemize}

\section{Scope}
Het eindresultaat is een werkend prototype. Dit prototype demonstreert de volledige ondertekenflow met een voorbeelddocument: een gebruiker kan het document inzien, ondertekent met een sms-code en het systeem legt deze actie vast in een audit trail. De opzet is generiek, zodat later elk type document ondersteund kan worden.

\section{Verwachte meerwaarde}
Deze module versnelt het klantenproces voor Energytix aanzienlijk. Het digitaliseren van overeenkomsten en andere documenten verlaagt de administratieve last en zorgt voor een snellere, rechtsgeldige activatie van diensten.

\section{Planning en werkbelasting}
\begin{itemize}
    \item Analyse \& opzet (15u)
    \item Backend ontwikkeling (40u)
    \item Frontend ontwikkeling (30u)
    \item Testen \& afronding (15u)
\end{itemize}